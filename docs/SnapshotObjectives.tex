\documentclass[12pt]{article}
\usepackage[margin=1in]{geometry}
\usepackage{titlesec}
\usepackage{lipsum}

\titleformat{\section}{\normalfont\Large\bfseries}{\thesection.}{1em}{}

\title{Snapshots}
\author{Jose Holguin, Andrew Wun, Alyssa Tu, Hyeon Seok Song}
\date{May 5, 2025}

\begin{document}

\maketitle

\section{Snapshot 1}
For our first objective we started the SRS and SDS with what our project will be needing in order to get started.
The development tools introduced will be Unity and Visual Studio code where we will use these softwares to create the functionality for the training simulator.
Since the goal is this project is to create a virtual training simulation, we will be using Blender and Autodesk maya for creating the models.
The models need to be designed realistic enough to simulate a real world experience.
For communication our group will be using discord and github. 
\subsection{Development Tools}
\begin{itemize}
    \item Unity 
    \item Microsoft Visual Studio Code
\end{itemize}
\subsection{3D Modeling}
\begin{itemize}
    \item Blender
    \item Autodesk Maya
\end{itemize}
\subsection{Communication}
\begin{itemize}
    \item Discord
    \item Github
\end{itemize}

\section{Snapshot 2}
In our second objective we started using Autodesk Maya and Blender for 3D modeling.
Our philosophy for the design around the models are meant to make it realistic.
We want the users to feel like they are in a real world. We used realistic photos of neighborhoods as reference.
Cosmetics aside, our other main objective was the pole master which is what controls the electrical wiring.
We had to make sure the polemaster is designed and functions close to a real life polemaster. We had to get wiring colors right as every
detail on the real life devices mattered.
\subsection{3D Models Created}
\begin{itemize}
    \item Trees
    \item Houses (backyard, porch)
    \item Sidewalks (streetlights, electrical poles)
    \item Polemaster (colors, lights, LEDS for different states)
    \item Smart Navigator (tool used for keeping track of electricity flow)
\end{itemize}

\section{Snapshot 3}
\subsection{Implementing 3D models functionality with Unity}
For this objective, after creating the models, we needed to give them functionality. That is where Unity comes in. 
Unity is used in our design specs as a game engine in which developers can create 2D, 3D, and VR games. This tool 
can use our model assets in order to give them the real life simulation effect. 
\subsection{Unity implementations}
\begin{itemize}
    \item User Interface and Menu Screen (3 buttons with each being a different task)
    \item Smart Navigator installation (steps for installing and clamping it onto the hotstick)
    \item Teleportation (for the user to immediately go to the task location)
    \item Inventory screen (allows user to store equipment)
\end{itemize}

\section{Snapshot 4}
\subsection{Final Touches and Future Implementations}
Lastly, for this final section, we were able to test and implement extra features like audio and menu task. 
For future features we would like to add more of SCE's training missions.
We also would have liked to add difficulty options for missions to give harder or easier variants of task.
\subsection{Extra Features}
\begin{itemize}
    \item Audio Effects when certain criterias are met (correct completions of task)
    \item Menu Task Screen (screen to check of completed objectives like gathering equipment, preparation)
\end{itemize}

\end{document}
