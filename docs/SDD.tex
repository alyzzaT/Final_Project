\documentclass[12pt]{article}
\usepackage[a4paper, margin=1in]{geometry}
\usepackage{fancyhdr}
\usepackage{hyperref}
\usepackage{longtable}
\usepackage{graphicx}

% Header and Footer
\pagestyle{fancy}
\fancyhf{}
\rhead{Software Design Document}
\rfoot{\thepage}

% Title Info
\title{\textbf{Software Design Document (SDD)} \\ \vspace{0.5cm} \large Southern California Edison Virtual Reality Training Program}
\author{Jose Holguin, Alyssa Tu, Andrew Wun, Hyun Seok Song}
\date{April 1, 2025}

\begin{document}

% Cover Page
\maketitle
\newpage

% Table of Contents
\tableofcontents
\newpage

% Versions Table
\section*{Version History}
\begin{longtable}{|p{1.5cm}|p{2cm}|p{3.5cm}|p{2cm}|}
\hline
\textbf{Name} & \textbf{Date} & \textbf{Revisions} & \textbf{Version} \\
\hline
Group 7 & May 2, 2025 & Snapshot 1 (materials needed and first draft/section 1) & Version 1.0 \\
\hline
Group 7 & May 3, 2025 & Snapshot 2 (added model design details) & Version 1.1 \\
\hline
Group 7 & May 4, 2025 & Snapshot 3 (unity implementation details) & Version 1.2 \\
\hline
Group 7 & May 5, 2025 & Snapshot 4 (final product and added future implementation ideas) & Version 1.3 \\
\hline
% Add more as needed
\end{longtable}
\newpage

% Sections

\section{Introduction}
\subsection{Purpose}
The purpose of this software is to help train Southern California Edision(SCE) 3rd party contractors. 
The software is called Virtual Reality Training Application(VRTA).
It is an interactive training simulation software for Southern California Edison. 
The initial issue given at hand was that SCE sources contractors not trained under their practices. 
This software will help address the issue by training them with the skills from SCE.

\subsection{Intended Audience}
This document is intended for software developers, advisors, and SCE sponsors who need an overview and understanding of our project. 
The table of contents has the sections for what specifics the reader would like to know about of our project. 
Section 2 goes more in depth over the interface of our project and how the user will interact with the software.

\subsection{Overview}
Our first objective in this document is to layout the 
materials needed, frameworks, and dependencies to structure our project.
Our second objective in this project was to start working on the models using Autodesk Maya and Blender.
The third objective was Unity implementation and adding functionality to the models for the virtual reality.
Lastly we went over our finished product, talked about extra features added like audio, and went over
what future features could be added if we continued to work on it.

\section{System Architecture}
\subsection{Overview of Architecture}
The two main componens of this software is Unity Engine and Blender.
The Unity engine is a development platform that allows users to create gaming programs in 2D and 3D, while also 
being able to mimic physics. We had to create our models to use in this program software
so thats where Blender comes in which allows users to create 3D models, simulation renders, or even
animations. Blender was used to create the specific SCE equipment and "world" 
that the training would be working in. 

\subsection{Workflow and Site Breakdown}
Since our second objective is designing models, we spent time looking around in real world environments and talking talking 
to SCE employees for designing these models. We wanted to make sure we were creating a close to real training program.
Afterwards we needed Unity for the VR interaction functionality.
The user would control their own movements and the models used.
The objects the user would control would be the equipment and the hands-on task.
The user would then have to control the buttons to move around. Since teleportation was used for movement,
that makes button pressing the preferred execution as it is easy and trivial. A software called Oculus
allows developers to view whats being seen through their computer, 
this would grant easier access to what the user is doing when testing. Our last objective was adding extra features like
scripts, audios, and extra assets. All of our assets and audio files are stored in the Unity projects folder.


\section{User Interface}
\subsection{UI}
\begin{enumerate}
    \item Main Menu
    \begin{itemize}
        \item Scenes
        \item Restart
        \item Task
        \item Buttons (Previous/Next)
    \end{itemize}

    \item Inventory
    \begin{itemize}
        \item Slot
        \item UI
        \item Dropped Items
        \item Controls
    \end{itemize}
\end{enumerate}

\subsection{Database Explanation}
Model assets, audio clips, and scripts are stored within the project itself on Unity.
As game starts and the user is put into the virtual reality with the use of a Meta Quest headset.
All models are grabbed from the files and loaded up into the game world. Specific visuals and audios
are put in place when there is a trigger. Thats when the scripts come in and implement these updates dynamically.

\section{Glossary}
\begin{longtable}{|p{4cm}|p{10cm}|}
\hline
\textbf{Term} & \textbf{Definition} \\
\hline
UI & User Interface \\
SCE & Southern California Edison \\
VRTA & Virtual Reality Training Application \\
% Add more as needed
\hline
\end{longtable}

\section{References}
\begin{thebibliography}{9}
    \bibitem{bliz2022}
    Bliz Studio.  
    \textit{Unity VR XR Interaction Toolkit Grabbable UI Canvas Tablet} [Video].  
    YouTube, September 26, 2022.  
    \url{https://www.youtube.com/watch?v=TCixKyOGTRU}
    
    \bibitem{bliz2022}
    Bliz Studio.  
    \textit{Unity VR XR Interaction Toolkit Grabbable UI Canvas Tablet} [Video].  
    YouTube, September 26, 2022.  
    \url{https://www.youtube.com/watch?v=TCixKyOGTRU}

    \end{thebibliography}

\end{document}
