\documentclass[12pt]{article}
\usepackage[a4paper, margin=1in]{geometry}
\usepackage{fancyhdr}
\usepackage{graphicx}
\usepackage{amsmath}
\usepackage{longtable}
\usepackage{hyperref}

\pagestyle{fancy}
\fancyhf{}
\rhead{\thepage}
\lhead{Software Requirements Specification}

\title{Software Requirements Specification (SRS)\\[1em] \large for Southern California Edison VR Training Application}
\author{Jose Holguin, Andrew Wun, Alyssa Tu, Hyun Seok Song}
\date{May 1, 2025}

\begin{document}

% Cover Page
\maketitle
\thispagestyle{empty}
\newpage

% Table of Contents
\tableofcontents
\newpage

% Version Table
\section*{Revision History}
\begin{longtable}{|c|c|c|c|}
\hline
\textbf{Name} & \textbf{Date} & \textbf{Reason for Changes} & \textbf{Version} \\
\hline
Group 7 & May 1, 2025 & Started draft/section1 with Snapshot 1 specs & 1.0 \\
\hline
Group 7 & May 3, 2025 & started snapshot 2 model design requirements & 1.1 \\
\hline
Group 7 & May 4, 2025 & snapshot 3 unity implementation & 1.2 \\
\hline
Group 7 & May 5, 2025 & snapshot 4 finished project and future implementations & 1.3 \\
\hline
% Add more rows as needed
\end{longtable}
\newpage

% SECTION 1 - INTRODUCTION
\section{Introduction}
The Virtual Reality Training Application(VRTA) is an interactive training simulation software for Southern California Edison.
The document will give an analysis and breakdown of the features and functionalities implemented. An explanation of the 
tools and frameworks needed for this projects completion is given as well.

\subsection{Purpose}
The purpose of this software is to help train Southern California Edision(SCE) 3rd party contractors. 
The initial issue given at hand was that SCE sources contractors not trained under their practices. 
This software will help address the issue by training them with the skills from SCE.
The document will also talk about our design philosophy for models using Blender and how we
used Unity to create this virtual reality.

\subsection{Intended Audience}
This document is intended for software developers, advisors, and SCE sponsors who need an overview and understanding of our project.
The table of contents has the sections for what specifics the reader would like to know about of our project.
Section 2 goes mre over the interface of our project and how the user will interact with the software.

\subsection{Overview}
The software product is a VR training simulator that uses Unity which is a software application that is mostly used for video game creation.
We created models to mimic SCE equipment and real life training simulations. For the designs we wanted to be as realistic as we could, 
so we interviewed SCE workers for specifics.
The software engine Unity offers the tools to help create modeling renders and functionality for the training application. 
The overview is that the person being trained will get to be in an interactive digital environment that will mimic real life working conditions.
This is in order to prepare the trainee's for how SCE does their own contracting.


% SECTION 2 - EXTERNAL INTERFACES
\section{External Interface Requirements}
The VRTA application provides a 3D graphical user interface (GUI). 
The layout and design follow virtual reality practices to ensure a consistent and intuitive experience for users.

\subsection{User Interfaces}
When designing the user interface, we wanted to use a streamlined tool like Blender to help create an easy flow.
The menu uses models that are easy to understand when moving around and picking task.
The user with a homescreen that has 3 buttons for navigation. 
The buttons are Underground Training, Overhead Training, and Quit.
Each of the training buttons will redirect the user to a task title scene of the requested work environment
Task title scene will present user with title, brief overview, estimated time of completion, and difficulty level.
There will also be two buttons: Back and Next.
The chosen task will display instructions, contain button to go back to previous step, a progress bar showing how much of task was completed, 
replicate a real world underground-onsite setting or overhead-onsite setting.
The application will give user feedback based off buttons pressed in the task like an error message for incorrect inputs.
Each step has a button that will allow you to continue once the task has been completed correctly.
Once the task is completed, the scene will provide a prompt of completion and two buttons:
Home, Quit,

\subsection{Software Interfaces}
\begin{itemize}
    \item 3.3.1 Android-based operating system version 8.0 or higher
    \item 3.3.2 Unity version 2021.3.5f1
    \item 3.3.3 Meta Quest 2 version 46 or higher
    \item 3.3.3 3.3.4 Windows 10 or higher
\end{itemize}


% SECTION 3 - LEGAL AND ETHICAL
\section{Legal and Ethical Considerations}
\subsection{Data Storage and Privacy}
There is no database storage but as the software doesn't require the user to put any private information.
SCE is a public utility company that has doesn't privatize their skill requirement so the application can display these training simulations.
Models and audio files are stored in the Unity project folder itself and are rendered when certain actions within the virtual world
are completed. 

\subsection{Ethical Concerns}
The software Application only mimics the actual training, it does not accurately depict real life training.
This technology is meant to help for when working on real utilities but will not give the trainee proper understanding.
The user is not meant to rely soley on this technology, it is a guiding tool.

% SECTION 4 - GLOSSARY
\section*{Glossary}
\begin{longtable}{|p{4cm}|p{10cm}|}
\hline
\textbf{Term/Acronym} & \textbf{Definition} \\
\hline
SCE & Southern California Edison \\
GUI & Graphical User Interface \\
VRTA & Virtual Reality Training Application \\
\hline
\end{longtable}

\end{document}
